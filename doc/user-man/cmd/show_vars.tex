% -*-latex-*-
\begin{nusmvCommand}{show\_vars} {Shows model's symbolic variables and defines with their types}

\cmdLine{show\_vars [-h] [-s] [-f] [-i] [-t | -V | -D] [-v] [-m | -o
    output-file]}

Prints a summary of the variables and defines declared in the input
file. Moreover, it prints also the list of symbolic input, frozen and
state variables of the model with their range of values (as defined in
the input file) if the proper command option is specified.

By default, if no type specifiers (\commandopt{s}, \commandopt{f},
\commandopt{i}) are used, all variable types will be printed. When
using one or more type specifiers (e.g. \commandopt{s}), only
variables belonging to selected types will be printed.

\begin{cmdOpt}

\opt{-s}{Prints only state variables.}

\opt{-f}{Prints only frozen variables.}

\opt{-i}{Prints only input variables.}

\opt{-t}{Prints only the number of variables (among selected kinds),
  grouped by type. This option is incompatible with \commandopt{V} or
  \commandopt{D}}

\opt{-V}{Prints only the list of variables with their types (among
  selected kinds), with no summary information. This option is
  incompatible with \commandopt{t} or \commandopt{D}}

\opt{-D}{Prints only the list of defines with their types, with no
  summary information. This option is
  incompatible with \commandopt{t} or \commandopt{V}}

\opt{-v}{Prints verbosely. Scalar variable's values are not truncated
  if too long for printing.}

\opt{-m}{Pipes the output to the program specified by the
\shellvar{PAGER} shell variable if defined, else through the
\unix command \shellcommand{more}.}

\opt{-o \parameter{\filename{output-file}}}{Writes the output generated by the command to
\filename{output-file}.}

\end{cmdOpt}

\end{nusmvCommand}
